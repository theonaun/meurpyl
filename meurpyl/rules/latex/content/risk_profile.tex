\section{Risk Profile}
\thispagestyle{section_start_style}

	\subsection{Introduction}

		\begin{tcolorbox}
			This section covers many basic details of the Python language, along with rudimentary descriptions of various risk types; it may not be necessary for all readers.
		\end{tcolorbox}

		Python has a number of unique characteristics that makes it unlike other languages within the enterprise space. Among other things, it is interpreted instead of compiled, it uses syntatically significant whitespace instead of braces, and it uses a virtual machine specific to the Python language. These attributes significantly affect the risks Python presents and the manner in which they can be mitigated.

		This section will examine:

		\begin{enumerate}
   			\item Interpretation
   			\begin{enumerate}
	   			\item Intentional misuse
	   			\item Unintentional misuse
	   			\item Model risk
	   			\item Governance risk
	   			\item Maturity risk
   			\end{enumerate}
   			\item Syntax
   			\item Dynamic tyling
   			\item Standard library
   			\item Packages
		\end{enumerate}

	\subsection{Interpretation}
		The largest difference between Python and other enterprise languages from a risk management perspective is that Python can be thought of as an interpreted language, as opposed to a compiled language.
		\footnote{\url{https://en.wikipedia.org/wiki/Interpreted_language}}
		% Footnote comma sep
		$^{,}$
		\footnote{\url{https://en.wikipedia.org/wiki/Compiled_language}}
		Computing languages can fall on a spectrum of those that are primarily interpreted (such as R and Python) and those that are primarily compiled (such as COBOL and C++). The difference is that compiled languages do a comparatively larger portion of any computation at the time of compilation, and have minimal runtime environments. Interpreted languages do most or all of the computations required at runtime. Though Python is partially interpreted and partially compiled, for the purposes of these rules Python will be referenced as an interpreted language.

		The risk dynamic with interpreted languages is inherently different from compiled languages. With compiled languages, it is relatively easy segregate the development and source code of an application from the user of an application. A development group can compile the application's source code, and the resulting binary can be forwarded to a completely separate environment for the end user. It is exceedingly difficult for an end user to modify the compiled binary and cause unintended behavior. With interpreted languages, the end user of a program is essentially being given possession of the application's source code. A sophisticated end-user could modify this source code cause unintended behavior. Arguably, one could say that this is tantamount to the development environment being included on the end users computer.

		\subsubsection{Intentional misuse}
			One major source of risk with development in any computing language is intentional misuse. This misuse can manifest itself in any number of ways, such as through the theft of customer information, embezzlement, or myriad other negative consequences.


		\iffalse
			Using foreign packkages people who don't know.

			operations reputational
			The distinction between interpreted and compiled becomes important when 
			Managing risk with compiled languages within the enterprise is accomplished by segregating the 
			The addition of an interpreter and run time ...
			Every environment is arguably a development environment ...

			distinction
			Compiled languages are comparatively easy to secure.
			The distinction between interpreted and compiled becomes 
			All languages share certain risks.
			The use of any languages brings risk. 
		\fi
			
	\subsection{Dynamic Typing}

		Typing

	\subsection{Syntax}

		Accessability is a double edged sword.

	\subsection{Standard Library}

		%Can do things, unlike COBOL.

		T%ESTING?

		%https://ithandbook.ffiec.gov/media/274741/ffiec_itbooklet_developmentandacquisition.pdf

		%https://ithandbook.ffiec.gov/media/274793/ffiec_itbooklet_informationsecurity.pdf

		%https://ithandbook.ffiec.gov/media/274825/ffiec_itbooklet_operations.pdf

		%https://ithandbook.ffiec.gov/media/274809/ffiec_itbooklet_management.pdf

		%ISO 9000 standards?

		%https://ithandbook.ffiec.gov/media/274709/ffiec_itbooklet_audit.pdf