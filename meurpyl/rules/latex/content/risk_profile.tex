\section{Risk Profile}

\thispagestyle{section_start_style}

	\subsection{Introduction}

		\begin{tcolorbox}
			This section covers many basic details of the Python language, along with rudimentary descriptions of various risk types; it may not be necessary for all readers.
		\end{tcolorbox}

		Python has a number of unique characteristics that makes it unlike other languages within the enterprise space. Among other things, it is interpreted instead of compiled, it uses syntatically significant whitespace instead of braces, and it uses a virtual machine specific to the Python language. These attributes significantly affect the risks Python presents and the manner in which they can be mitigated.

	\subsection{Interpretation}

		The largest difference between Python and other enterprise languages from a risk management perspective is that Python can be thought of as an interpreted language, as opposed to a compiled language.\footnote{\url{https://en.wikipedia.org/wiki/Interpreted_language}} Computing languages can fall on a spectrum of those that are primarily interpreted (such as R and Python) and those that are primarily compiled (such as COBOL and C++). The difference is that compiled languages do a comparatively larger portion of any computation at the time of compilation, and have minimal runtime environments.\footnote{\url{https://en.wikipedia.org/wiki/Compiled_language}} Interpreted languages do most or all of the computations required at runtime. Though Python is partially interpreted and partially compiled, for the purposes of these rules Python will be referenced as an interpreted language.

		The risk dynamic with interpreted languages is inherently different from compiled languages. With compiled languages, it is relatively easy segregate the development and source code of an application from the user of an application. A development group can compile the application's source code, and the resulting binary can be forwarded to a completely separate environment for the end user. It is exceedingly difficult for an end user to modify the compiled binary and cause unintended behavior. With interpreted languages, however, the end user of a program is essentially being given possession of the application's source code. The risks involved are more akin to a shell than a basic application due to Python's inherent flexibility. A sophisticated end-user could modify this source code or create his or her own source code to cause unintended behavior. While this is not appreciably different than having access to a shell such as Powershell or Bash, it is a distinct risk.

		\subsubsection{Intentional misuse}

			One major source of risk with development in any computing language within enterprise is intentional misuse. This misuse can manifest itself in any number of ways. Some individuals skirt rules and use tools in an unintended manner to make their lives easier. Others may misuse tools to enrich themselves or to hurt their companies.Some examples of intentional misuse are:

			\textbf{Theft, Embezzlement, and Misappropriation.}
			Because the individual controlling a Python interpreter has the ability to perform most abilities that a user can conduct manually, Python can be used to collect and process large amounts of information. Absent controls, Python can be used to scan large amounts of data, process it, and transfer it in an automated manner. This makes shells, interpreted languages, and similar tools attractive employees looking to engage in the theft of customer information or intellectual property.

			\textbf{Unsanctioned Development.}
			With the Python language, it is not possible to limit the creation of programs to groups with access to compilers. To run Python programs, users require Python interpreters. If users have Python interpreters, they necessarily have the tools needed to create programs. If employees have the ability to develop software, there is a non-negligible chance that they will do so. These non-development employees are often the least equipped to address software development risks due to their lack of familiarity with the process. In the financial services context, this also covers the unsanctioned development of models, which is governed by Office of the Comptroller of the Currency ("OCC") guidance.\footnote{\url{https://www.occ.gov/news-issuances/bulletins/2011/bulletin-2011-12.html}}

			\textbf{Impermissible Use.}
			In addition to illegal activities and unsanctioned devleopment, intentional misuse can take the form of ...


		\subsubsection{Unintentional misuse}
			
	\subsection{Dynamic Typing}

	\subsection{Syntax}

	\subsection{Standard Library}

	\subsection{Language Support}