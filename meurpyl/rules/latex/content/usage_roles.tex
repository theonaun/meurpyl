\section{Usage Roles}

\thispagestyle{section_start_style}

	\subsection*{Introduction}

		The rules and controls that should exist for a Python are determined by the risks and benefits involved. These risks and benefits, in turn, are derived from: a) the inherent attributes of the Python language, and b) the manner in which the Python language is used. The inherent attributes of the Python lanugage are largely fixed, but the manner in which Python is use is anything but.

		An organization will likely have a multiplicity of use cases for Python encompassing everything from applications programming to data analysis. These use cases can conceptually be treated as a spectrum or as a group of discrete types of use cases. Both approaches have benefits and drawbacks--a spectrum sacrifices administrative efficiency because each use case must be approached individually; discrete groups of use cases sacrifice granularity because each use case must be shoehorned into a use case that it may or may not be suited to.

		For the purposes of these rules, discrete groups of use cases is demonstrably the better choice. Usage categories can be written in such a way as to cover broad areas of usage that share similar charactaristics. At the same time, each and every use case permutation need not be captured. Items that do not fall within the criteria for inclusion in any other category can be shunted into a specialized cateogry for individualized treatment.

		MEURPyL imposes five arbitrary Usage Roles ("UR"):

		\begin{enumerate}
        	\item Data Exploration (DE)
        	\item Data Processing (DP)
        	\item System Adminstration (SA)
        	\item Systems Development (SD)
        	\item Unclassified Usage (UU)
        \end{enumerate}

%%%%%%%%%%%%%%%%%%%%%%%%%%%%%%%%%%%%%%%%%%%%%%%%%%%%%%%%%%%%%%%%%%%%%%%%%%%%%%

	\subsection{Data Exploration (DE)}

		\subsubsection{Objective}

			From a risk management perspective, using Python for data analysis is indistinguishable from using Python for application development. No matter the purpose, the users is programming and will have access to the same powerful standard library that has the potential for misuse or abuse. At the same time, the iterative nature of data analysis is unsuited to the typical controls around application development lifecycles. 

			By segregating certain Python analytical activities within a Data Exploration usage role, it is possible to create an environment that strikes a balance between control and flexibility.

		\subsubsection{Description}

			Due to its ease-of-use and the powerful data libraries availible, Python has become the primary language used in data science. Data science and its associated activities have unique requirements that warrant a specialized usage category. 

			Exploratory Data Analysis ("EDA") is an integral part of any data analysis workflow.\footnote{\url{https://en.wikipedia.org/wiki/Exploratory_data_analysis}} This process allows for analysts to become familiar with production data, generate pre-models to inform the model risk management process, and glean insights through trial and error with that data. 

			EDA involves both programming and production analysis, which often have mutually exclusive controls--consequently, it is ill-suited to being completely in either domain. The common thread with EDA activities is that they: a) take production data, b) are nimble processes with minimal overhead or risk, and c) output information or other ephemeral deliverables that other groups can use to create non-EDA activities.

		\subsubsection{Criteria}

			In order to qualify for inclusion under this environment, Python usage must:

			\begin{enumerate}
        		\item Have end deliverables confined to low-risk, non-mission-critical activities
        		\item Be used in a non-automated manner, with human intervention in place for all processing
        		\item Have analysis or programs in final form used ten or fewer times over the course of a program's lifetime
        		\item When running programs, the programs should be run by the program's creator
        	\end{enumerate}

			In order to qualify for inclusion under this environment, Python usage must not:

			\begin{enumerate}
        		\item Directly impact account balances or other money movement
        		\item Access Personally Identifiable Information ("PII")
        		\item Payment Card Industry ("PCI") Primary Account Number ("PAN") Information
        		\item Move significant amounts of data outside the organization
        		\item Create models outside a model risk management framework
        	\end{enumerate}

		\subsubsection{Examples}

			\begin{enumerate}
        		\item Interactive workstation-based analysis on public or low-risk data
        		\item Multi-user Jupyter notebook server-based analysis on public or low-risk data
        		\item Research and the creation of pre-models for input into a model risk mnagement framework
        	\end{enumerate}


%%%%%%%%%%%%%%%%%%%%%%%%%%%%%%%%%%%%%%%%%%%%%%%%%%%%%%%%%%%%%%%%%%%%%%%%%%%%%%

	\subsection{Data Processing (DP)}

		\subsubsection{Objective}

			Due to its flexibility and interpreted nature, Python makes it easy to move from proof-of-concept to production. Consequently, it is also easy for individuals to create their own processes outside established channels. These unauthorized processes may not have sufficient validation controls, unit tests, or permissions.

			By providing a standardized Data Processing usage role with clearly defined requirements, it is possible to minimize the number of individuals establishing uncontrolled processing environments, while minimizing the overhead involved.

		\subsubsection{Description}

			A

		\subsubsection{Criteria}

			In order to qualify for inclusion under this environment, Python usage must:

			\begin{enumerate}
        		\item A
        	\end{enumerate}

			In order to qualify for inclusion under this environment, Python usage must not:

			\begin{enumerate}
        		\item A
        	\end{enumerate}

		\subsubsection{Examples}

%%%%%%%%%%%%%%%%%%%%%%%%%%%%%%%%%%%%%%%%%%%%%%%%%%%%%%%%%%%%%%%%%%%%%%%%%%%%%%

	\subsection{Systems Administation (SA)}

		\subsubsection{Objective}

			A

		\subsubsection{Description}

			A

		\subsubsection{Criteria}

			In order to qualify for inclusion under this environment, Python usage must:

			\begin{enumerate}
        		\item A
        	\end{enumerate}

			In order to qualify for inclusion under this environment, Python usage must not:

			\begin{enumerate}
        		\item A
        	\end{enumerate}

		\subsubsection{Examples}

			A

%%%%%%%%%%%%%%%%%%%%%%%%%%%%%%%%%%%%%%%%%%%%%%%%%%%%%%%%%%%%%%%%%%%%%%%%%%%%%%

	\subsection{Systems Development (SD)}

		\subsubsection{Objective}

			A

		\subsubsection{Description}

			A

		\subsubsection{Criteria}

			In order to qualify for inclusion under this environment, Python usage must:

			\begin{enumerate}
        		\item A
        	\end{enumerate}

			In order to qualify for inclusion under this environment, Python usage must not:

			\begin{enumerate}
        		\item A
        	\end{enumerate}

		\subsubsection{Examples}

			A

%%%%%%%%%%%%%%%%%%%%%%%%%%%%%%%%%%%%%%%%%%%%%%%%%%%%%%%%%%%%%%%%%%%%%%%%%%%%%%

	\subsection{Unclassified Usage (UU)}

		\subsubsection{Objective}

			A

		\subsubsection{Description}

			As

		\subsubsection{Criteria}

			In order to qualify for inclusion under this environment, Python usage must:

			\begin{enumerate}
        		\item A
        	\end{enumerate}

			In order to qualify for inclusion under this environment, Python usage must not:

			\begin{enumerate}
        		\item A
        	\end{enumerate}

		\subsubsection{Examples}

			A
