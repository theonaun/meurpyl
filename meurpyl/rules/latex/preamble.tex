
% TODO@TN: Make less preachy and more informal.
% TODO@TN: Peer review for copyediting and content.
% TODO@TN: Use fewer idioms.

\subsection*{Preamble}
\thispagestyle{blank_style}
% No subsection prior to this, so subsection is blank

Corporations are creatures of habit; they do today what they did yesterday (because what they did yesterday worked). Convention has its benefits: it is stable, predictable, and even comforting. It is not completely without cost, however. Companies that fail to innovate stagnate and die every day.

Enterprise Python shares these trade offs. Unlike COBOL, it does not have fifty years of enterprise usage. Unlike C\#, it is not backed up with an army of software engineers and support staff to integrate the language within existing intrastructure. In its place, Python provides a fluid and expressive environment that is unparalelled for rapid development and data science.

Risk managers and information technology employees often recognize these strengths, but are hesistant to fully accept Python in their organizations without fuller understanding of the risks involved. In order to make Python more palatable at the enterprise level, some tradeoffs must be made. A measure of autonomy is sacrificed for the sake of control and oversight. Some flexibility is sacrificed for the sake of stability and uniformity. Empowerment is sacrificed for the sake of security and mitigating risk. 

These rules use the following guiding principles to inform these trade-offs:

\begin{itemize}
	\item \textbf{The correct path should be the path of least resistance}--employees are more likely to comply if the rules make it easy to do so.
	\item \textbf{Arbitrary is better than indefinite}--having bright line rules reduces confusion even if it does not always yield the ideal outcome.
	\item \textbf{Make it as safe as it needs to be (but no safer)}--things cannot be made completely safe, but they can be made completely useless.
\end{itemize}

Finally, the Model Enterprise Usage Rules for the Python Language ("MEURPyL" or "Meurpyl") provides a model for a set of baseline standards governing the Python language. Meurpyl is just that: a baseline. It is not dogma that is intended to be adopted without question. Rather, it is intended to function as a base by which organizations can adapt Python for their own purposes--using what works and ignoring the rest.
