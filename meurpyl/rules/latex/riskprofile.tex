\section{Risk Profile}

	\subsection{Introduction}

		\begin{tcolorbox}
			This section covers many basic details of the Python language, along with rudimentary descriptions of various risk types; it may not be necessary for all readers.
		\end{tcolorbox}

		Python has a number of unique charactaristics that makes it unlike other languages within the enterprise space. Among other things, it is interpreted instead of compiled, it uses syntatically significant whitespace instead of braces, and it uses a virtual machine specific to the Python language. These attributes significantly affect the risks Python presents and the manner in which they can be mitigated.

		This section will examine:

		\begin{enumerate}
   			\item Interpretation
   			\begin{enumerate}
	   			\item Intentional misuse
	   			\item Unintentional misuse
	   			\item Model risk
	   			\item Governance risk
   			\end{enumerate}
   			\item Syntax
   			\item Dynamic tyling
   			\item Standard library
   			\item Packages
		\end{enumerate}

	\subsection{Interpretation}
		The largest difference between Python and other enterprise languages from a risk management perspective is that Python can be thought of as an interpreted language, as opposed to a compiled language.
		\footnote{
			\url{
				https://en.wikipedia.org/wiki/Interpreted_language
			}
		}
		% Footnote comma sep
		$^{,}$
		\footnote{
			\url{
				https://en.wikipedia.org/wiki/Compiled_language
			}
		}
		Computing languages can fall on a spectrum of those that are primarily interpreted (such as R and Python) and those that are primarily compiled (such as COBOL and C++). The difference is that compiled languages do a comparatively larger portion of any computation at the time of compilation, and have minimal runtime environments. Interpreted languages do most or all of the computations required at runtime. Though Python is partially interpreted and partially compiled, for the purposes of these rules Python will be referenced as an interpreted language.

		One major source of risk with any computing language is intentional misuse. This misuse can manifest itself through the theft of customer information, embezzlement, or myriad other negative consequences.

		With compiled languages, this is risk is partially mitigated by maintaining a strict separation of testing and production environments. The individuals who control an application by writing and compiling code have minimal control over how that code is used in production.

		distinction
		Compiled languages are comparatively easy to secure.
		The distinction between interpreted and compiled becomes 
		All languages share certain risks.
		The use of any languages brings risk. 

		\subsubsection{Intentional misuse}
			A

		Using foreign packkages people who don't know.

		operations reputational
		The distinction between interpreted and compiled becomes important when 
		Managing risk with compiled languages within the enterprise is accomplished by segregating the 
		The addition of an interpreter and run time ...
		Every environment is arguably a development environment ...

			
	\subsection{Dynamic Typing}

		Typing

	\subsection{Syntax}

		Accessability

	\subsection{Standard Library}

		Can do things, unlike COBOL.