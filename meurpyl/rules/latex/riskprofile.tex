\section{Risk Profile}

\subsection{Introduction}

\begin{tcolorbox}
	This section covers many basic details of the Python language; it may not be necessary for all readers.
\end{tcolorbox}

Python is unlike other enterprise languages. For example, it is interpreted instead of compiled, it uses syntatically significant whitespace instead of braces, and it uses a virtual machine specific to the Python language. 

\subsection{Interpretation}
	The largest difference between Python and other enterprise languages from a risk management perspective is that Python can be thought of as an interpreted language, as opposed to a compiled language.
	\footnote{
		\url{
			https://en.wikipedia.org/wiki/Interpreted_language
		}
	}
	% Footnote comma sep
	$^{,}$
	\footnote{
		\url{
			https://en.wikipedia.org/wiki/Compiled_language
		}
	}

	Computing languages can fall on a spectrum of those that are primarily interpreted (such as R and Python) and those that are primarily compiled (such as COBOL and C++). The difference is that compiled languages do a comparatively larger portion of any computation at the time of compilation, and have minimal runtime environments. Interpreted languages do most or all of the computations required at runtime. Both approaches have their advantages, and like other languages, Python is both compiled language and interpreted.

	Managing risk with compiled languages ...

	The addition of an interpreter and run time ...

	
\subsection{Dynamic Typing}

	Typing

\subsection{Syntax}

	Accessability

\subsection{Standard Library}

	Can do things, unlike COBOL.